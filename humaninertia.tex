% Template for PLoS
% Version 1.0 January 2009
%
% To compile to pdf, run:
% latex plos.template
% bibtex plos.template
% latex plos.template
% latex plos.template
% dvipdf plos.template

\documentclass[10pt]{article}

% amsmath package, useful for mathematical formulas
\usepackage{amsmath}
% amssymb package, useful for mathematical symbols
\usepackage{amssymb}

% graphicx package, useful for including eps and pdf graphics
% include graphics with the command \includegraphics
\usepackage{graphicx}

% cite package, to clean up citations in the main text. Do not remove.
\usepackage{cite}

\usepackage{color}

% TODO temporary
\usepackage{todonotes}

% Use doublespacing - comment out for single spacing
%\usepackage{setspace}
%\doublespacing


% Text layout
\topmargin 0.0cm
\oddsidemargin 0.5cm
\evensidemargin 0.5cm
\textwidth 16cm
\textheight 21cm

% Bold the 'Figure #' in the caption and separate it with a period
% Captions will be left justified
\usepackage[labelfont=bf,labelsep=period,justification=raggedright]{caption}

% Use the PLoS provided bibtex style
\bibliographystyle{plos2009}

% Remove brackets from numbering in List of References
\makeatletter
\renewcommand{\@biblabel}[1]{\quad#1.}
\makeatother


% Leave date blank
\date{}

\pagestyle{myheadings}
%% ** EDIT HERE **


%% ** EDIT HERE **
%% PLEASE INCLUDE ALL MACROS BELOW

%% END MACROS SECTION

\begin{document}

% Title must be 150 characters or less
\begin{flushleft}
{\Large
\textbf{Title}
}
% Insert Author names, affiliations and corresponding author email.
\\
Author1$^{1}$, 
Author2$^{2}$, 
Author3$^{3,\ast}$
\\
\bf{1} Author1 Dept/Program/Center, Institution Name, City, State, Country
\\
\bf{2} Author2 Dept/Program/Center, Institution Name, City, State, Country
\\
\bf{3} Author3 Dept/Program/Center, Institution Name, City, State, Country
\\
$\ast$ E-mail: Corresponding author@institute.edu
\end{flushleft}

% Please keep the abstract between 250 and 300 words
\section*{Abstract}

% Please keep the Author Summary between 150 and 200 words
% Use first person. PLoS ONE authors please skip this step. 
% Author Summary not valid for PLoS ONE submissions.   
\section*{Author Summary}

\section*{Introduction}

\subsection*{Yeadon's method}

\todo{Talk about the number of measurements and the number of joint angles}

\section*{Software design}

The input to \verb+yeadon+ consists of (1) measurements of a subject, and (2)
the joint configuration of the subject. With these two inputs, one is able to
obtain the mass properties of the subject.
The mass properties consist of the mass, center of mass, and moment of inertia
tensor. These properties can be obtained for the entire subject, or for
individual limbs of the subject. TODO any allowable configuration.

% Results and Discussion can be combined.
\section*{Results}

\subsection*{Subsection 1}

\subsection*{Subsection 2}

\section*{Discussion}

% You may title this section "Methods" or "Models". 
% "Models" is not a valid title for PLoS ONE authors. However, PLoS ONE
% authors may use "Analysis" 

\section*{Usage}

We demonstrate how one may use the \verb+yeadon+ package using the example of
an ice skater performing a spin. As is commonly taught in high school physics
classes when discussing the conservation of angular momentum, an ice skater
extends her arms to spin slowly, and draws them proximally to increase her
angular velocity. Since the product $I_{zz}\omega$ remains constant and $I_{zz}$
decreases, then $\omega$ must increase (with the $z$ axis directed vertically).
We can use this human inertia model to vary a model's configuration to quickly
observe the effect on $I_{zz}$. Specifically, we want to know the factor by
which an ice skater's angular velocity may increase when she draws her arms in
from a directly outward position.

We begin by describing how \verb+yeadon+ is obtained and installed, then
discuss the two ways that \verb+yeadon+ can be used to approach this problem.

\subsection*{Obtaining and installing the software}

The package, along with its documentation, is distributed through the Python
Package Index. Accordingly, the package can be downloaded and installed easily
from the command line on a user's machine using the \verb+pip+ package:

\begin{verbatim}
$ pip install yeadon
\end{verbatim}

There are two ways to use the package: through methods of the \verb+Human+
class, or through an interactive command-line user interface. We cover both.

\subsection*{The Human class}

\begin{verbatim}
import yeadon as y
h = y.Human('female1.txt')
\end{verbatim}

This constructs the \verb+Human+ with the segment dimensions as given in the
\verb+female1.txt+ input file. This human has the default configuration, in
which all joint angles are zero. This human can be visualized by executing the
following:

\begin{verbatim}
h.draw()
\end{verbatim}
\todo{is draw() correct}

Figure \ref{fig:femaledefault} shows this Human

TODO show the use of some other methods, for instance averaging and scaling.

\subsection*{The command-line user interface}


-dev for python2.7
-assume unix machine
-dist through pip
-how to install
-two ways to use.


\todo{Measurement figure}

% Do NOT remove this, even if you are not including acknowledgments
\section*{Acknowledgments}


%\section*{References}
% The bibtex filename
\bibliography{library}

\section*{Figure Legends}
%\begin{figure}[!ht]
%\begin{center}
%%\includegraphics[width=4in]{figure_name.2.eps}
%\end{center}
%\caption{
%{\bf Bold the first sentence.}  Rest of figure 2  caption.  Caption 
%should be left justified, as specified by the options to the caption 
%package.
%}
%\label{Figure_label}
%\end{figure}


\section*{Tables}
%\begin{table}[!ht]
%\caption{
%\bf{Table title}}
%\begin{tabular}{|c|c|c|}
%table information
%\end{tabular}
%\begin{flushleft}Table caption
%\end{flushleft}
%\label{tab:label}
% \end{table}

\end{document}

